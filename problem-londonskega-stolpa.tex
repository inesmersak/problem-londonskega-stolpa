\documentclass[12pt,a4paper]{amsart}
% ukazi za delo s slovenscino -- izberi kodiranje, ki ti ustreza
\usepackage[slovene]{babel}
%\usepackage[cp1250]{inputenc}
%\usepackage[T1]{fontenc}
\usepackage[utf8]{inputenc}
\usepackage{amsmath,amssymb,amsfonts}
\usepackage{url}
%\usepackage[normalem]{ulem}
\usepackage[dvipsnames,usenames]{color}

% ne spreminjaj podatkov, ki vplivajo na obliko strani
\textwidth 15cm
\textheight 24cm
\oddsidemargin.5cm
\evensidemargin.5cm
\topmargin-5mm
\addtolength{\footskip}{10pt}
\pagestyle{plain}
\overfullrule=15pt % oznaci predlogo vrstico


% ukazi za matematicna okolja
\theoremstyle{definition} % tekst napisan pokoncno
\newtheorem{definicija}{Definicija}[section]
\newtheorem{primer}[definicija]{Primer}
\newtheorem{opomba}[definicija]{Opomba}

\theoremstyle{plain} % tekst napisan posevno
\newtheorem{lema}[definicija]{Lema}
\newtheorem{izrek}[definicija]{Izrek}
\newtheorem{trditev}[definicija]{Trditev}
\newtheorem{posledica}[definicija]{Posledica}


% za stevilske mnozice uporabi naslednje simbole
\newcommand{\R}{\mathbb R}
\newcommand{\N}{\mathbb N}
\newcommand{\Z}{\mathbb Z}
\newcommand{\C}{\mathbb C}
\newcommand{\Q}{\mathbb Q}


% naslednje ukaze ustrezno popravi
\newcommand{\program}{Matematika} % ime studijskega programa: Matematika/Finan"cna matematika
\newcommand{\imeavtorja}{Ines Meršak} % ime avtorja
\newcommand{\imementorja}{prof.~dr. Sandi Klavžar} % akademski naziv in ime mentorja
\newcommand{\naslovdela}{Problem londonskega stolpa}
\newcommand{\letnica}{2016} %letnica diplome


% commands
\newcommand{\graf}[1]{\ensuremath{#1 = (V(#1), E(#1))}}

% operatorji
\DeclareMathOperator {\stopnja} {deg}




\begin{document}

% od tod do povzetka ne spreminjaj nicesar
\thispagestyle{empty}
\noindent{\large
UNIVERZA V LJUBLJANI\\[1mm]
FAKULTETA ZA MATEMATIKO IN FIZIKO\\[5mm]
\program\ -- 1.~stopnja}
\vfill

\begin{center}{\large
\imeavtorja\\[2mm]
{\bf \naslovdela}\\[10mm]
Delo diplomskega seminarja\\[1cm]
Mentor: \imementorja}
\end{center}
\vfill

\noindent{\large
Ljubljana, \letnica}
\pagebreak

\thispagestyle{empty}
\tableofcontents
\pagebreak

\thispagestyle{empty}
\begin{center}
{\bf \naslovdela}\\[3mm]
{\sc Povzetek}
\end{center}
% tekst povzetka v slovenscini
V povzetku na kratko opi"si vsebinske rezultate dela. Sem ne sodi razlaga organizacije dela -- v katerem poglavju/razdelku je kaj, pa"c pa le opis vsebine.
\vfill
\begin{center}
{\bf The Tower of London problem}\\[3mm] % prevod slovenskega naslova dela 
{\sc Abstract}
\end{center}
% tekst povzetka v anglescini
Prevod zgornjega povzetka v angle"s"cino.

\vfill\noindent
{\bf Math. Subj. Class. (2010):} navedi vsaj eno klasifikacijsko oznako -- dostopne so na \url{www.ams.org/mathscinet/msc/msc2010.html}  \\[1mm]  
{\bf Klju"cne besede:} navedi nekaj klju"cnih pojmov, ki nastopajo v delu  \\[1mm]  
{\bf Keywords:} angle"ski prevod klju"cnih besed
\pagebreak


 
% tu se zacne tekst seminarja
\section{Uvod}
Test londonskega stolpa je ena izmed variacij Hanojskih stolpov. Izumil ga je britanski nevropsiholog Tim Shallice leta 1982. Pogosto je uporabljen v psihologiji, saj s pomočjo te igre ugotavljajo stanje pacientove psihe, opazujejo pa lahko tudi napredek bolezni pri npr.\ Parkinsonovih bolnikih.

Osnovna verzija londonskega stolpa vsebuje tri enako velike krogle različnih barv in tri palice. Na prvo palico lahko postavimo samo eno kroglo, na drugo le dve krogli, na tretjo pa tri. Cilj igre je priti iz nekega danega stanja v neko drugo želeno stanje s čim manj koraki.

(Viri: wikipedia (TS), KlavžarHinz knjiga)

Vsa možna stanja in prehode med njimi lahko zelo elegantno opišemo s pomočjo grafov, zato si najprej poglejmo nekaj osnovnih pojmov teorije grafov.

\section{Osnovni pojmi teorije grafov}

\begin{definicija}
	\emph{Graf} $G$ je urejen par $(V(G), E(G))$, kjer je $V(G)$ končna množica \emph{vozlišč}, $E(G)$ pa množica \emph{povezav} grafa. Povezave so predstavljene kot neurejeni pari vozlišč (neusmerjeni grafi).
\end{definicija}

Obstajajo variacije zgornje definicije, graf je lahko npr.\ usmerjen (povezave so usmerjeni pari) -- tedaj govorimo o \emph{digrafih}, ima neskončno število vozlišč ali pa več povezav med dvema vozliščima.

Vozlišča grafa predstavimo s točkami v ravnini, povezavo med dvema vozliščima pa kot enostavno krivuljo med ustreznima točkama v ravnini.

\begin{definicija}
	Če je $e = \{ u,v \}$ povezava, tedaj sta $u$ in $v$ \emph{krajišči} povezave $e$, pišemo tudi $e = uv$; rečemo, da sta $u$ in $v$ \emph{sosednji vozlišči}.
	
	\emph{Soseščina} vozlišča $u$ je množica 
	\[ N(u) = \{ x\colon ux \in E(G) \} .\]
	\emph{Stopnja vozlišča} $u$ je število vseh vozlišč, ki so mu sosednji: $\stopnja u = |N(u)|$.
\end{definicija}

\begin{definicija}
	Graf $\graf{H}$ je \emph{podgraf} grafa $\graf{G}$, če velja 
	\[ V(H) \subseteq V(G) \text{ in } E(H) \subseteq E(G). \]
	Podgraf H grafa G je \emph{inducirani} podgraf, če za vsaki dve vozlišči $x,y\in V(H)$ velja: $xy \in E(G) \implies xy \in E(H)$.
\end{definicija}

\begin{definicija}
	\emph{Pot} na $n$ vozliščih je graf, ki ima dva vozlišča stopnje $1$, medtem ko so preostala vozlišča stopnje $2$. TODO
	
	\emph{Cikel} na $n$ vozliščih (označimo ga s $C_n$) je graf, ki ga dobimo iz poti na $n$ vozliščih tako, da dodamo povezavo med vozliščema stopnje $1$.
\end{definicija}

\begin{primer}
	Sem pride primer poti in cikla za $n=6$ (recimo).
\end{primer}

\begin{definicija}
	Naj bosta $\graf{G}$ in $\graf{H}$ grafa. 
	Preslikava $f\colon V(G) \longrightarrow V(H)$ je \emph{izomorfizem}, če je bijektivna in velja
	\[ uv \in E(G) \iff f(u)f(v) \in E(H),\quad \forall u, v \in E(G). \]
	Grafa $G$ in $H$ sta \emph{izomorfna}, če obstaja izomorfizem $G \longrightarrow H$. Oznaka: $G \cong H$.
\end{definicija}

\begin{definicija}
	\emph{Pot v grafu} $G$ je podgraf grafa $G$, ki je izomorfen poti.
	\emph{Cikel v grafu} $G$ je podgraf grafa $G$, ki je izomorfen ciklu. 
\end{definicija}

\begin{definicija}
	\emph{Sprehod} v grafu $G$ je zaporedje vozlišč $v_1, v_2, \ldots, v_k$, tako da velja $v_i v_{i+1} \in E(G),\ 1 \leq i \leq k-1$. Če označimo $x = x_1$ in $y = x_k$, potem takemu sprehodu rečemo $x,y$-sprehod.

	Sprehod je \emph{enostaven}, če so vsa njegova vozlišča različna. Enostaven sprehod inducira podgraf, ki je izomorfen poti.
\end{definicija}

\begin{definicija}
	Naj bo $G$ graf in $\sim$ relacija, definirana na $V(G) \times V(G)$:
	\[ x \sim y \ \stackrel{\text{def}}{\equiv} \ \exists \ x,y \text{-sprehod.} \]
\end{definicija}

Preprosto lahko preverimo, da je relacija $\sim$ ekvivalenčna. Torej sledi:

\begin{definicija}
	Relacija $\sim$ množico vozlišč grafa G $V(G)$ razbije na ekvivalenčne razrede. Podgrafi, inducirani s temi razredi, so \emph{komponente} grafa $G$.
	
	Graf je \emph{povezan}, če ima le eno komponento.
\end{definicija}

\begin{definicija}
	Naj bo $G$ povezan graf. Tedaj je \emph{razdalja} $d_G(u,v)$ med vozliščema $u$ in $v$ najmanjše možno število povezav na neki $u,v$-poti.
\end{definicija}

\begin{definicija}
	Pot v grafu, ki vsebuje vsa vozlišča tega grafa, se imenuje \emph{Hamiltonova pot}.
	\emph{Hamiltonov cikel} nekega grafa $G$ je cikel v $G$, ki poteka skozi vsa vozlišča tega grafa.
	Graf je \emph{Hamiltonov}, če vsebuje Hamiltonov cikel.
\end{definicija}

\begin{izrek}
	Naj bo $G$ Hamiltonov graf. Za vsako podmnožico vozlišč $X \subseteq V(G)$ velja, da ima graf $G - X$ kvečjemu $|X|$ komponent.
\end{izrek}

\proof
	Blabla tukaj je dokaz. Blabla tukaj je dokaz. Blabla tukaj je dokaz. Blabla tukaj je dokaz. Blabla tukaj je dokaz.
\endproof

\begin{definicija}
	Graf je \emph{ravninski}, če ga lahko narišemo v ravnini tako, da se nobeni povezavi ne križata.
\end{definicija}

\begin{definicija}
	\emph{Barvanje} grafa $\graf{G}$ je preslikava 
	\[ c\colon V(G) \longrightarrow \N, \text{ tako da velja } av \in E(G) \implies c(a) \neq c(v). \]
	Če je $c\colon V(G) \longrightarrow [k]$, rečemo, da je $c$ $k$-barvanje. \emph{Kromatično število} grafa $G$, $\mathcal{X}(G)$, je najmanjši $k$, za katerega obstaja $k$-barvanje grafa G.
\end{definicija}

% seznam uporabljene literature
\begin{thebibliography}{99}

%\bibitem{}

\end{thebibliography}

\end{document}

